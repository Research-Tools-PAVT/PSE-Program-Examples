\section{Experimental Evalutaion}
\label{sec:experiments}

\sumit{What experiments and examples to work on?}
\sumit{In 4 pages}

\subsection{(RQ1) \baseaxp vs. \tool}
\begin{itemize}
	\item We run \axp with default settings where it samples the values of the forall variables at random and simulates multiple runs of the program with different values of the probabilistic variables for each sampled value of the forall variables.
	\item \tool is also run with default settings, the values of the forall variables are determined by solving the path constraints via symbolic execution using \klee and for each setting of the forall variables, the program is run multiple times with different values of the probabilistic variables.
\end{itemize}
\subsection{(RQ2) Hybrid Strategy}
We run \tool under a hybrid strategy here. Symbolic Execution is used to find the worst case forall settings that lead to the violation of the probabilistic assert in the program. For each of the worst case setting found using symbolic execution, \axp runs the program multiple times over different values of the probabilistic variables. Any error detected by \axp is considered a violation of the probabilistic assert.