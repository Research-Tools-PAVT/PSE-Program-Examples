\documentclass{article}

\usepackage{algorithm, algpseudocode}

\algnewcommand{\algorithmicforeach}{\textbf{for each}}
\algdef{SE}[FOR]{ForEach}{EndForEach}[1]
{\algorithmicforeach\ #1\ \algorithmicdo}% \ForEach{#1}
{\algorithmicend\ \algorithmicforeach}% \EndForEach
\usepackage{minted} 

\title{PSE Quant Sampling Algorithm}
\author{Sumit Lahiri}

\begin{document}
\maketitle

We try to formulate a way to compute path probabilities using \texttt{symbolic execution} and \texttt{testing} based technique.
\begin{minted}{c} 
int main(void)
{
	int a; // unintialized
	int d = std::uniform_distribution<rd_seed>(0, 650);
	
	// forall variable : (INT_MIN to INT_MAX)
	klee_make_symbolic(&a, sizeof(a), "a_sym");
	
	// PSE variable : Uniformly distributed [0 to 650]
	make_pse_symbolic<int>(&d, sizeof(d), "d_prob_sym", 0, 650);
	
	int c = a + 100;
	
	// case 1 : Pure Forall Predicate
	if (a > 50) {
	  c = a + 75;
	} else {
	  c = a - 75;
	}
	
	// case 2 : Pure PSE Predicate
	if (d > 60) d = 250;
	
	// case 3 : Dependence Case
	if (c > d) c = d;

	// Probabilistic query : assert(P(c != d) < 0.5)
	// Optimize here : 
	//	Optimal value of forall (a) such that P(c != d) is close to 0.5 	
	return 0;
}
\end{minted}

\begin{algorithm}[H]
	\caption{Candidates : (Testing Based Estimation)}
	\begin{algorithmic}[1]
		\ForEach{$p \in Paths$}%
		\State $c := ConstraintSet(p)$ \algorithmiccomment{Path Constraints for p}
		\State $m := Optimize(query,  c)$ \algorithmiccomment{solution for the path constraints}
		\State $concreteSet = \{ \}$
		\ForEach{$v \in ForallVars(p)$} \algorithmiccomment{ForallVars p $\rightarrow $ forall} %
		\State $concreteSet.append(\{key : v, val : m[v]\})$ \algorithmiccomment{Candidate Values}
		\EndForEach
		\State $executeCV(program, concreteSet)$
		\EndForEach
	\end{algorithmic}
\end{algorithm}

\begin{algorithm}[H]
	\caption{executeCV : PSE Sampled Normal Execution}
	\begin{algorithmic}[1]
		\Function{executeCV}{$P : program, C : concreteSet$}
		\ForEach{$v \in ForallVars(p)$}%
		\State value(v) := concreteSet(v) \algorithmiccomment{Use values from ConcreteSet}
		\EndForEach
		\State ... \algorithmiccomment{proceed with normal execution}
		\EndFunction
	\end{algorithmic}
\end{algorithm}

For the sample program given above, we first resort to using \texttt{symbolic execution} to generate \texttt{path constraints} for all the feasible paths that this program can take and then convert the \texttt{path constraints} into an \texttt{formal logic} optimization problem that gives an \texttt{assignment} to \texttt{forall} variables such that it leads to optimum violation of the $query$.

\begin{minted}{python}
def generateCandidates(k: int): # Candidates Algorithm
	opt = z3.Optimize()
	a = z3.Int("a_sym")
	d = z3.Int("d_prob_sym")
	
	opt.add(d >= 0)
	opt.add(d <= 650)
	opt.add(a > 50)
	opt.add(z3.Not(d > 60))
	opt.add(a + 75 > d)

	opt.maximize(a - d - 75) 	# Query to optimize
	n = 0
	while opt.check() == z3.sat and n != k:
		m = opt.model()
		n += 1
		print("%s = %s" % (a, m[a]))
		print("%s = %s" % (d, m[d]))
		opt.add(a != m[a])
\end{minted}

We now explore a slightly different example which is more involved in terms of the constraints and query that the user can pose at the end of the \texttt{symbolic execution}.

\begin{minted}{c}
	int a, b, c, d;
	
	// forall variables : (INT_MIN to INT_MAX)
	klee_make_symbolic(&a, sizeof(a), "a_sym");
	klee_make_symbolic(&b, sizeof(b), "b_sym");
	klee_make_symbolic(&c, sizeof(c), "c_sym");
	
	// PSE variables
	make_pse_symbolic<int>(&d, sizeof(d), "d_prob_sym", 0, 500);
	
	// PSE variable : Random Sampling
	std::default_random_engine generator;
	std::uniform_int_distribution<int> distribution(0, 500);
	
	if (a + b > c + d)
	{
		if (a > b) {
			a = 100;
			b = 500;
		} else {
			a = 500;
			b = 100;
		}
	} else {
		if (c > d) {
			a = 100;
			c = 100;
			b = 600;
			d = distribution(generator);
		} else {
			a = 600;
			c = 600;
			b = 100;
			d = distribution(generator);
		}
	}

	if (a + c > b + d)
	{
		d = distribution(generator);
	}
\end{minted}

Below we show a sample of the type of the \texttt{queries} that a user can make in the context of the example shown above. 

\begin{minted}{c}
	assert(a + b + c + d <= 1100);
	
	Case : 1
	[Query Parse] : P(a + b + c + d <= 1100) >= 0.5
	Query : assert fails atleast half of the times.
	
	Case : 2
	[Query Parse] : P(a + b + c + d <= 1100) <= 0.5
	Query : assert fails atmost half of the times.
	
	Case : 3
	[Query Parse] : P(a + b + c + d <= 1100) == 0.5
	Query : assert fails exactly half of the times.
\end{minted}

Based on the query posed, we get \texttt{candidate} models by converting the code into a \texttt{optimization} query and solve this optimization problem using any off-the-shelf 
\texttt{SMT Solver} to get model values for the forall variables that contribute to maximum 
violation of the \texttt{query} constraint.

\begin{minted}{c}
	Path : [And(d_prob_sym >= 0, d_prob_sym <= 500), 
	a_sym + b_sym > c_sym + d_prob_sym, 
	a_sym > b_sym, 500 + d_prob_sym < 100 + c_sym]
		Model : 1
			a_sym = 402
			b_sym = 0
			c_sym = 401
		Model : 2
			a_sym = 404
			b_sym = -1
			c_sym = 402
	Path : [And(d_prob_sym >= 0, d_prob_sym <= 500), 
	a_sym + b_sym > c_sym + d_prob_sym, 
	a_sym > b_sym, Not(500 + d_prob_sym < 100 + c_sym)]
		Model : 1
			a_sym = 1
			b_sym = 0
			c_sym = 0
		Model : 2
			a_sym = 0
			b_sym = -1
			c_sym = -2
	Path : [And(d_prob_sym >= 0, d_prob_sym <= 500), 
	a_sym + b_sym > c_sym + d_prob_sym, 
	Not(a_sym > b_sym), 100 + d_prob_sym < 500 + c_sym]
		Model : 1
			a_sym = -199
			b_sym = -199
			c_sym = -399
		Model : 2
			a_sym = -200
			b_sym = -197
			c_sym = -398
	Path : [And(d_prob_sym >= 0, d_prob_sym <= 500), 
	a_sym + b_sym > c_sym + d_prob_sym, 
	Not(a_sym > b_sym), Not(100 + d_prob_sym < 500 + c_sym)]
		...
\end{minted}

Here \texttt{P(condition)} represents the sum path \texttt{probabilities} for a given \texttt{condition} to hold with some deterministic value as posed in the queries shown above.  

We now do a \texttt{transformation} pass over the program and instrument the count of the given \texttt{condition} failing. In the \texttt{transforamtion} pass, we make the \texttt{program} take values that we find as \texttt{candidate} models upon following \texttt{Algorithm 1}

\begin{minted}{c}
	int a, b, c, d, assert_satisfy = 0, termCount = 35000;
	
	// forall variables take values from stdin.
	// pass values that conform to a candidate model
	scanf("%d", &a); // (INT_MIN to INT_MAX)
	scanf("%d", &b); // (INT_MIN to INT_MAX)
	scanf("%d", &c); // (INT_MIN to INT_MAX)
	
	while(termCount--) {
		...
		(code remains same)
		(run the code in a while loop)
		...
		// Added by transformation pass
		// We take a count of how many times the 
		// posed query is satisfied.
		if (a + b + c + d - 1100 <= 0) {
			assert_satisfy++;
		}
	}
\end{minted}

\end{document}