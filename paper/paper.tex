% This is samplepaper.tex, a sample chapter demonstrating the
% LLNCS macro package for Springer Computer Science proceedings;
% Version 2.20 of 2017/10/04
%
\documentclass[runningheads]{llncs}
%\renewcommand\UrlFont{\color{blue}\rmfamily}

\usepackage[nounderscore]{syntax}
\usepackage{bussproofs}
\usepackage{mathtools}
\usepackage{bibnames}
\usepackage[ruled,vlined,linesnumbered,noend]{algorithm2e}
\usepackage{amsmath}
\usepackage{amsfonts}
\usepackage{amssymb}
\usepackage{graphicx}
\usepackage{longtable}
\usepackage{supertabular}
\usepackage{booktabs}
\usepackage{minted}
\usepackage{colortbl}
\usepackage{multirow}
\usepackage{xcolor}
\usepackage{comment} 
\usepackage{enumitem} 
\usepackage{tabularx} 

\newcommand{\sr}[1]{\textcolor{red}{Subhajit: #1}}
\newcommand{\sumit}[1]{\textcolor{blue}{Sumit: #1}}
\newcommand{\tab}[1]{Table~\ref{#1}}
\newcommand{\fig}[1]{Figure~\ref{#1}}
\newcommand{\code}[1]{Listing~\ref{#1}}

\definecolor{bg}{RGB}{210, 255, 180}
\definecolor{aquamarine}{rgb}{0.5, 1.0, 0.83}
\usepackage{subcaption} %% For complex figures with subfigures/subcaptions
%% http://ctan.org/pkg/subcaption
\usepackage{mathtools}

\newcommand*\E[1]{\mathbb{E}\left[ #1 \right]}
\newcommand{\sem}[1]{\llbracket #1 \rrbracket}
\definecolor{lg}{gray}{0.82}
\newcommand{\instantiate}[1]{#1}
\newcommand{\baseaxp}{\textsc{AxProf (Baseline)}\xspace}
\newcommand{\axp}{\textsc{AxProf}\xspace}
\newcommand{\tool}{\textsc{PSETool}\xspace}
\newcommand{\klee}{\textsc{KLEE}\xspace}
\newcommand{\var}[1]{\texttt{#1}}
\newcommand{\highlight}[1]{\colorbox{pink}{#1}}
\newcommand{\highlightcode}[1]{\colorbox{bg}{#1}}
\newcommand{\highlightflag}[1]{\colorbox{yellow}{#1}}
\newcommand{\colorprog}[1]{\colorbox{aquamarine}{#1}}
\newcommand{\colorpred}[1]{\colorbox{green}{#1}}
\newcommand{\colorinv}[1]{\colorbox{yellow}{#1}}
\newcommand{\coloralg}[1]{\colorbox{lg}{#1}}

\setlist[description]{leftmargin=10pt,labelindent=5pt,topsep=2pt} 
\setlist[itemize]{leftmargin=15pt,labelindent=0pt,topsep=2pt} 
\setlength{\intextsep}{10pt} % Vertical space above & below [h] floats
\setlength{\textfloatsep}{5pt} % Vertical space below (above) [t] ([b]) floats
\setlength{\floatsep}{5pt} % Vertical space below (above) [t] ([b]) floats
\setlength{\abovecaptionskip}{5pt}
\setlength{\belowcaptionskip}{5pt}
%\renewcommand{\cite}[1]{\citep{#1}}

\begin{document}
%
\title{Probabilistic Symbolic Execution\thanks{We submit an online anonymous docker image of our tool along with a ZIP file containing all experimental artifacts and scripts for regeneration.}}
%
%\titlerunning{Abbreviated paper title}
% If the paper title is too long for the running head, you can set
% an abbreviated paper title here
%
%\author{First Author\inst{1}\orcidID{0000-1111-2222-3333} \and
%Second Author\inst{2,3}\orcidID{1111-2222-3333-4444} \and
%Third Author\inst{3}\orcidID{2222--3333-4444-5555}}
%
%\authorrunning{F. Author et al.}
% First names are abbreviated in the running head.
% If there are more than two authors, 'et al.' is used.
%
%\institute{Princeton University, Princeton NJ 08544, USA \and
%Springer Heidelberg, Tiergartenstr. 17, 69121 Heidelberg, Germany
%\email{lncs@springer.com}\\
%\url{http://www.springer.com/gp/computer-science/lncs} \and
%ABC Institute, Rupert-Karls-University Heidelberg, Heidelberg, Germany\\
%\email{\{abc,lncs\}@uni-heidelberg.de}}
%
\maketitle              % typeset the header of the contribution
%
\begin{abstract}
\label{sec:abstract}
The abstract should briefly summarize the contents of the paper in
150--250 words

\keywords{Probabilisitic Programming \and Symbolic Execution \and Logic \and Verification.}
\end{abstract}

We thank the anonymous reviewers for their valuable comments. We provide some clarifications
to their queries and comments in this document. 

This document is structured in 3 sections:

\begin{itemize}
	\item Section 1 (Overview)
	\item Section 2 (Detailed Comments)
	\item Section 3 (Conclusion)
\end{itemize}

\input{parts/overview.tex}
\section{Algorithm}
\label{sec:algorithm}

\sumit{Layout the Algorithm text and Algorithm in package} 
\sumit{In 4 pages}

\begin{algorithm}
	\caption{Probabilistic Symbolic Execution Algorithm}
	\label{alg:symb_ex}
	\SetAlgoLined
	\small
	%\KwResult{result}
	% \While{true}{
	set the program state to $\vec{s}$\;
\end{algorithm}
% !TeX root = ../paper.tex
\section{Implementation}
\label{sec:implementation}

We build on top of KLEE~\cite{klee_main_2008}, use Z3~\cite{z3_solver_2008} for solving the SMT formula for computing the $\E{v}$ values
\section{Experimental Evalutaion}
\label{sec:experiments}

\sumit{What experiments and examples to work on?}
\sumit{In 4 pages}

\begin{itemize}
	\item (RQ1) Experiments
	\item (RQ2) Additional Examples
\end{itemize}
\input{parts/related_works.tex}
\input{parts/conclusion.tex}

\bibliographystyle{splncs04}
\bibliography{refs}
%
\end{document}
